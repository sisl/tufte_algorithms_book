\chapter{Introduction}
\label{ch:introduction}

Welcome to the book template!

\todo{This is something that I still need to do.}

\tim{This is a comment that Tim inserted.}
\mykel{And this is Mykel. Go ahead and add new commands associated with your own names by editing \texttt{book.cls}.}

This template is based on our previous book.\cite{Kochenderfer2019}

\section{Using Julia}

In this section we provide some examples of how Julia algorithms can be typeset, tested, etc.
We begin with a Julia algorithm, \cref{alg:sample_function}.
We use our custom \texttt{algorithm} environment with a pythontex \texttt{juliaverbatim} environment inside to typeset the algorithm.
A caption provides some additional information, and shows up in the margin.

\begin{algorithm}
\begin{juliaverbatim}
function sample_function(x, a)
	if x > a
		return log(x)
	else
		return x + log(a) - a
	end
end
\end{juliaverbatim}
\caption{
	\label{alg:sample_function}
	A sample function that takes in an evaluation scalar \jlv{x} and a scalar parameter \jlv{a}.
}
\end{algorithm}

We can add a test for our algorithm in the source code.
This test does not show up when you compile.
You can run all tests by executing \texttt{jl runtests.jl}.

\begin{juliatest}
let
	@test isapprox(sample_function(1.0, 2), -0.30685, atol=1e-5)
	@test isapprox(sample_function(1.0, 3), -0.90139, atol=1e-5)
	@test isapprox(sample_function(1.5, 2), -0.19315, atol=1e-5)
end
\end{juliatest}

The code can be executed when creating figures.
\Cref{fig:sample_function} is a standard inline figure that executes Julia code using \jlpkg{PGFPlots} to produce a \TeX file in the \texttt{\\fig} directory.
This file is then compiled into the PDF.

\begin{figure}
	\begin{jlcode}
	p = let
		xs = collect(range(0.0,stop=10.0,length=101))
		plots = Plots.Plot[]
		for a in [1,2,3,5]
			push!(plots,
				Plots.Linear(xs, [sample_function(x, a) for x in xs],
				   style="solid, thick, mark=none", legendentry="parameter \$a = $a\$"))
		end
		Axis(plots, width="8cm", height="5cm", xlabel=L"x", ylabel="sample function output",
					style="cycle list name = pastelcolors",
		 		    legendPos="outer north east")
	end
	plot(p)
	\end{jlcode}
	\begin{center}
		\plot{fig/sample_function}
	\end{center}
	\caption{
		\label{fig:sample_function}
		Curves obtained when using \jlv{sample_function} for several different values of \jlv{a}.
	}
\end{figure}

Of course, there is nothing stopping you from inserting tikzpictures directly.
\begin{figure}
	\centering
	\begin{tikzpicture}[x=1cm, y=1cm]
		\draw[pastelBlue] (0,0) rectangle (2,2);
		\fill[pastelRed]  (3,-1) rectangle ++(2,1);
		\fill[pastelPurple]  (7,0.5) circle (0.5);
		\node[anchor=south] at (4,0) {red rectangle};
	\end{tikzpicture}
	\caption{
		\label{fig:sample_tikzpicture}
		A figure made using the \texttt{tikzpicture} environment.
	}
\end{figure}